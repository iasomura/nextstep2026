% =========================
% Memo: Base-rate adjustment (prior shift) for evaluation
% =========================

\subsection*{Base-rate sensitivity (prior shift) analysis}
\label{sec:base_rate_sensitivity}

\paragraph{Motivation.}
Our main test set is class-balanced (approximately 1:1). However, in real-world monitoring streams, phishing is typically rare.
Therefore, we estimate how Precision (PPV) and $F_1$ change under different class priors (base rates) \emph{without} resampling negatives,
using Bayes' rule and the confusion-matrix-derived rates.

\paragraph{Assumption (important).}
This analysis assumes \textbf{prior shift only}: the base rate $p=\Pr(y=1)$ changes while the conditional distributions (and thus $TPR$ and $FPR$)
remain the same. Under covariate shift (feature distribution changes), $TPR/FPR$ may also change and requires separate verification.

\paragraph{Notation.}
Let $TPR$ denote recall (true positive rate), and $FPR$ denote false positive rate:
\begin{align}
TPR &= \Pr(\hat{y}=1 \mid y=1) = \frac{TP}{TP+FN}, \\
FPR &= \Pr(\hat{y}=1 \mid y=0) = \frac{FP}{FP+TN}.
\end{align}
Given base rate $p=\Pr(y=1)$, the expected Precision (PPV) is:
\begin{equation}
PPV(p) \equiv \Pr(y=1 \mid \hat{y}=1)
= \frac{TPR \cdot p}{TPR \cdot p + FPR \cdot (1-p)}.
\label{eq:ppv_prior_shift}
\end{equation}
We compute $F_1$ as usual from $PPV(p)$ and $TPR$:
\begin{equation}
F_1(p) = \frac{2 \cdot PPV(p) \cdot TPR}{PPV(p) + TPR}.
\label{eq:f1_prior_shift}
\end{equation}

\paragraph{Key interpretation.}
\begin{itemize}
  \item Under prior shift, $TPR$ (Recall) does \emph{not} change with $p$; hence false negatives per positive remain stable.
  \item $PPV$ \emph{does} degrade as phishing becomes rarer because $FP$ scales with the negative mass $(1-p)$.
  \item For sufficiently small $p$, $FP \approx FPR \cdot N_{\text{total}}$ saturates (since $N_{\text{neg}} \approx N_{\text{total}}$).
\end{itemize}

\paragraph{Base-rate table (correct ratio direction).}
Using the full-cascade rates from Table~X (e.g., $TPR=0.9818$, $FPR=0.0084$), we obtain the following projections.
\textbf{Note:} the ratio is written as \textbf{Benign:Phish} (e.g., 10:1 means 10 benign per 1 phishing).
\begin{center}
\begin{tabular}{rccc}
\hline
Benign:Phish & $PPV$ (Precision) & $TPR$ (Recall) & $F_1$ \\
\hline
1:1    & 99.15\% & 98.18\% & 98.66\% \\
5:1    & 95.90\% & 98.18\% & 97.03\% \\
10:1   & 92.12\% & 98.18\% & 95.05\% \\
20:1   & 85.39\% & 98.18\% & 91.34\% \\
50:1   & 70.04\% & 98.18\% & 81.76\% \\
100:1  & 53.89\% & 98.18\% & 69.59\% \\
1000:1 & 10.46\% & 98.18\% & 18.91\% \\
\hline
\end{tabular}
\end{center}

\paragraph{Operational reading (alerts).}
For a daily volume $N_{\text{total}}$, expected false positives and true positives are:
\begin{align}
FP &\approx FPR \cdot (1-p) \cdot N_{\text{total}}, \\
TP &\approx TPR \cdot p \cdot N_{\text{total}}.
\end{align}
When $p$ is small, $(1-p)\approx 1$ and thus $FP \approx FPR \cdot N_{\text{total}}$ becomes almost independent of the base rate.
For example, with $FPR=0.0084$ and $N_{\text{total}}=10{,}000$, $FP \approx 84$ per day (and $\approx 42$ per day at $p=0.5$).

\paragraph{Reverse requirement (stronger, recommended).}
We can also \emph{invert} Eq.~\eqref{eq:ppv_prior_shift} to estimate the minimum base rate required to achieve a target precision $PPV^\star$:
\begin{equation}
p \ge
\frac{PPV^\star \cdot FPR}{TPR \cdot (1-PPV^\star) + PPV^\star \cdot FPR}.
\label{eq:prior_threshold_for_ppv}
\end{equation}
With $TPR=0.9818$ and $FPR=0.0084$, achieving $PPV^\star=0.90$ requires $p \gtrsim 0.0715$ (i.e., Benign:Phish $\lesssim 13:1$),
and $PPV^\star=0.95$ requires $p \gtrsim 0.139$ (Benign:Phish $\lesssim 6:1$).
This provides a quantitative rationale for \textbf{candidate-generation (Stage~0)}: pre-filtering must raise the effective base rate to a range
where high-precision operation is feasible.

\paragraph{Required FPR (dual reverse requirement).}
We can also solve Eq.~\eqref{eq:ppv_prior_shift} for $FPR$ to obtain the maximum tolerable false positive rate at a given base rate:
\begin{equation}
FPR \le \frac{TPR \cdot p \cdot (1-PPV^\star)}{PPV^\star \cdot (1-p)}.
\label{eq:fpr_threshold_for_ppv}
\end{equation}
With $TPR=0.9818$ and $PPV^\star=0.90$:
\begin{center}
\begin{tabular}{rcl}
\hline
Base rate $p$ & Benign:Phish & Required $FPR$ \\
\hline
0.5   & 1:1     & $\le 10.91\%$ \\
0.01  & 100:1   & $\le 0.110\%$ \\
0.001 & 1000:1  & $\le 0.0109\%$ \\
\hline
\end{tabular}
\end{center}
Our system's $FPR=0.84\%$ is adequate under balanced evaluation but exceeds the required level by an order of magnitude at CT-log scale.
This is not a weakness specific to our method---it is a structural constraint for \emph{any} detector with finite $FPR$ (base-rate fallacy).

\paragraph{Implication for CT-log-scale monitoring.}
If CT logs produce millions of benign certificates per day and only thousands are phishing (e.g., Benign:Phish $\approx 100{:}1$ to $1000{:}1$),
then even a small $FPR$ yields substantial false positives.
Our cascade is designed as a \textbf{classifier within a monitoring pipeline}, not as a standalone Internet-wide scanner.
Table~8 quantifies the base-rate condition that the upstream candidate-generation filter must satisfy.
In such settings, monitoring typically requires (i) a candidate filter to increase the effective base rate, and/or (ii) stricter thresholds to reduce $FPR$.

\paragraph{Caveats and next steps.}
Because CT-log benign distributions may differ from our benign test distribution, $FPR$ under CT logs may deviate from Table~X.
A follow-up evaluation should measure $FPR$ on a CT-like benign stream and report confidence intervals for $FPR$ and $PPV$.
